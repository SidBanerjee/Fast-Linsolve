\documentclass[letterpaper,conference]{ieeeconf}  
\IEEEoverridecommandlockouts
\overrideIEEEmargins

%\pagestyle{empty}

%\usepackage[small,compact]{titlesec}
\usepackage{amsmath,amssymb,amssymb,amsfonts,mathtools,dsfont}

\usepackage{comment}

%\usepackage{fullpage} %%%%%%%%%%%%%%%%%%%%% makes difference

\newcommand{\bigb}{\bigskip\noindent}
\newcommand{\med}{\medskip\noindent} 
\newcommand{\q}{\quad}
\newcommand{\sU}{\mathcal{U}}
\newcommand{\pd}{\partial}
\newcommand{\half}{\frac12} 
\newcommand{\RR}{{\mathcal R}}
\newcommand{\SSS}{{\mathcal S}}

\newtheorem{corollary}{Corollary}
\newtheorem{proposition}{Proposition}
\newtheorem{lemma}{Lemma}
\newtheorem{remark}{Remark}
\newtheorem{exercise}{Exercise}
\newtheorem{example}{Example}
\newtheorem{prop}{Proposition}
\newtheorem{theorem}{Theorem} 
\newtheorem{definition}{Definition} 
\newtheorem{infthm}{Informal Theorem}

\newlength{\figurewidth}
\setlength{\figurewidth}{5.5cm}% 
\newcommand{\diag}{\textrm{diag}}
\newcommand{\II}{\mathds{1}}
\newcommand{\vnorm}[1]{\left\|#1\right\|}


\def\EE{{\mathbb{E}}}
\def\PP{{\mathbb{P}}}
\def\BB{{\mathcal{B}}}
\def\MI{{\mathcal{I}}}
\def\setR{\mathbb{R}}
\def\setZ{\mathbb{Z}\xspace}
\def\setC{\mathbb{C}\xspace}
\def\setN{\mathbb{N}\xspace}
\def\setT{\mathbb{T}\xspace}
\def\setP{\mathbb{P}\xspace}
\def\setE{\mathbb{E}\xspace}
\def\bZ{\mathbf{Z}}
\def\bz{\mathbf{z}}
\def\FF{\mathcal{F}}
\def\mut{\mathcal{I}} 
\def\TT{{\mathcal T}}
\def\calE{{\mathcal{E}}}
\def\G{{\mathcal{G}}}
\def\E{{\mathcal{E}}}
\def\V{{\mathcal{V}}}

\newcommand{\qest}[1]{\mathbf{q}^{#1}_t}
\newcommand{\qestnew}[1]{\widetilde{\mathbf{q}}^{#1}_t}
\newcommand{\res}[1]{\mathbf{r}^{#1}_t}
\newcommand{\resnew}[1]{\widetilde{\mathbf{r}}^{#1}_t}
\newcommand{\lm}{\ell_{\max}}

\newcommand{\ip}[2]{\left< #1, #2 \right>}
\newcommand{\pzlt}{{\mathbf{p}}_\mathbf{z}^{\ell}[t]}
\newcommand{\pzltest}{\hat{\mathbf{p}}_\mathbf{z}^{\ell}[t]}

\newcommand{\Qstoch}[2]{q^s_{#1 #2}}
\newcommand{\Qsign}[2]{q^{sign}_{#1 #2}}
\newcommand{\pabsorb}[1]{[\mathbf{p}_A]_{#1}}

\usepackage{algorithm,algorithmic}

\renewcommand{\algorithmicrequire}{\textbf{Inputs:}}
\renewcommand{\algorithmicensure}{\textbf{Outputs:}}


\usepackage{color}
\usepackage{url}
\newcommand{\todo}{\color{red}}
\usepackage{color-edits}
%\usepackage[suppress]{color-edits}
\addauthor{ns}{blue}
\addauthor{sb}{magenta}

\usepackage[colorlinks,
			linkcolor=blue, 			
			citecolor=blue,
			urlcolor=magenta,
			linktocpage,
			plainpages=false,
			pdftex]{hyperref}



\begin{document}
 % or letter or a5paper or ... etc
% \geometry{landscape} % rotated page geometry

% See the ``Article customise'' template for come common customisations

\title{Sublinear Estimation of a Single Element in Sparse Linear Systems}

%%% BEGIN DOCUMENT

\author{Nitin Shyamkumar, Siddhartha Banerjee, Peter Lofgren\thanks{N. Shyamkumar is with the Department of Computer Science, and S. Banerjee is with the School of Operations Research and Information Engineering, Cornell University, USA, Email: {\tt nhs56@cornell.edu, sbanerjee@cornell.edu}. P. Lofgren is with the Department of Computer Science, Stanford University, USA, Email: {\tt plofgren@cs.stanford.edu}.}}

\maketitle

\begin{abstract}
We present a bidirectional algorithm for estimating a single element in the solution of a linear system $Ax=b$, with sublinear average-case running time guarantees for sparse systems.
Our work combines the von Neumann-Ulam scheme for solving linear systems with recent developments in bidirectional
algorithms for estimating random-walk metrics. 
In particular, given a target additive-error threshold, we show how to combine a local dynamic-programming iteration with a forward MCMC technique such that the resulting algorithm is order-wise faster than each individual
approach.      
\end{abstract}

%!TEX root = main.tex
\section{Introduction}
\label{sec:intro}

In this work, we develop a \emph{bidirectional algorithm} for estimating a \emph{single element} of the product of a matrix power and a vector.
Formally, given any matrix $A\in\setR^{n\times n}$, vector $\bz$ and \emph{a target index $t\in\{1,2,\ldots,n\}$}, we consider the problem of computing $\pzlt := \ip{A^{\ell}\bz}{\be_t}$ for given exponent $\ell\in\mathbb{N}$.

Computing $A^{\ell}\bz$ is a basic problem in matrix computations, and there is a large body of work on efficient ways of performing exact computation. 
However, in large-scale settings (i.e., with very large values of $n$), direct computation is often infeasible, and one needs to resort to estimating $A^{\ell}\bz$.
Moreover, we are interested in developing techniques which can estimate a single element of $A^{\ell}\bz$ in running time which is provably faster than computing the entire vector. 

The problem of estimating $\pzlt$ has gained attention recently in the context of estimating random-walk transition probabilities, in particular, for PageRank and Personalized PageRank~\cite{Page1999}, and other network centrality metrics. 
%Though PageRank had fast estimation algorithms based on iterative~\cite{Andersen2007} and MCMC~\cite{Avrachenkov2007} techniques, these 
A recent line of work~\cite{Lofgren2014,lofgren2016personalized,banerjee2015fast} has shown how to develop fast bidirectional algorithms for this problem, and more generally, for settings where $A^T$ is a \emph{stochastic} matrix and $\bz$ is a probability distribution (i.e., an element of the $n$-dimensional simplex).
Our work generalizes these techniques to more general $A,\bz$; in particular, our main result can be paraphrased as follows:
\begin{proposition}
Given matrix $A$ (with $||A||_1\leq 1$), vector $z$ (with $||z||_1\leq 1$) and any index $t$ in $[n]$, Algorithm \texttt{BIDIR-MATRIX-POWER} (cf. Section \ref{ssec:bidiralgo}) returns estimate $\pzltest$ such that with high probability $|\pzlt-\pzltest|<\max\{\epsilon\pzlt,1/n\}$, with average running time of $\widetilde{O}\left(\left(nnz(A)/\epsilon\right)^{2/3}\right)$ for uniform random choice of $t$.
\end{proposition}	

This result is interesting in that it generalizes the existing results for stochastic matrices to arbitrary matrices, albeit with a loss in the running time scaling~\footnote{For estimating $\ell$-step transition probabilities, the running time of an equivalent algorithm in ~\cite{banerjee2015fast} is $\widetilde{O}\left(\left(nnz(A)/\epsilon\right)^{1/2}\right)$ for uniform random choice of $t$.}. 
Moreover, computing $\left(A^{\ell}\bz\right)[t]$ is often a subroutine in more complex algorithms for various applications.
In Section \ref{sec:linsolve}, we focus on one such application -- the von Neumann-Ulam scheme for solving linear equations~\cite{forsythe1950matrix}. 
This has been gaining interest as a promising candidate for solving large-scale equations, owing to its ease of parallelization and asynchronous and local nature~\cite{ji2013convergence,dimov2015new,lee2014asynchronous}. 
More generally however, this algorithm may prove useful in more complex estimation tasks which use matrix polynomials as approximators. 


\subsection{Related Work}

Previous work has addressed both the problem of estimating a single component of a matrix equation and the problem of computing matrix inversions through Monte Carlo methods. The von Neumann-Ulam algorithm shows that the inverse of $B$ by defining $A = I - B$ and running random walks over the induced graph of $A$ \cite{forsythe1950matrix}. As long as the spectral norm of $A$ is less than $1$, the expectation of these random walks is exactly $(A)_{ij}$ where $i$ is the start node and $j$ is the end node. With these methods, one can solve the system $\mathbf{x} = G\mathbf{x} + \mathbf{z}$ provided the spectral norm $\rho(G) < 1$ since $\mathbf{x} = (I-G)^{-1}\mathbf{z}$, exactly the problem the von Neumann-Ulam algorithm solves. \cite{ji2013convergence} proved that there exist matrices $G$ satisfying $\rho(G) < 1$ but with $||G||_\infty > 1$, and that convergence of the von Neumann-Ulam algorithm is not guaranteed on this class of matrices.  Wu and Gleich address this issue by proposing a new algorithm that will converge as long as $\rho(G^+) < 1$, a weaker condition than $\rho(G) < 1$ \cite{wu2016multi}.

We can also view the problem of estimating a single component $\mathbf{x}[t]$ as computing the contributions in reverse from a source distribution to $\mathbf{x}[t]$.This problem was originally addressed for the special case of computing Personalized Page Rank (PPR) in \cite{andersen2007local} via the Reverse-Push algorithm, which pushes back the unit contribution using a series of reverse operations. \cite{andersen2007local} shows that this problem for the specific case of PPR can return an $\epsilon$ accurate estimate of $ppr(v)$ assuming $ppr(v) \geq \alpha$ in time $O\left( \frac{1}{\alpha \epsilon}\right)$.
This approach also resembles the work in \cite{lee2014asynchronous}, which describes a similar algorithm using an asynchronous procedure. \cite{lee2014asynchronous} performs a series of local updates using a residual vector, and show that with this procedure one can estimate $\mathbf{x}[t]$ satisfying $|\mathbf{x}[t] - \hat{\mathbf{x}[t]}| \leq \epsilon ||\mathbf{x}||$ in time $O\left(\min\{ \epsilon^{\ln d /\ln (|| G||_2)}, n \ln \epsilon / \ln (||G||_2)\} \right)$. 

Our work is based on recent developments in bidirectional algorithms for estimating single-state transition probabilities in Markov chains.
Such algorithms were first developed for \emph{reversible Markov chains} using random-walk collision statistics; in particular, Kale et al.~\cite{Kale2008} proposed such a technique for estimating length-$2\ell$ random walk transition probabilities in a \emph{regular undirected graph}.
The main idea is that to test if a random walk goes from $s$ to $t$ in $2\ell$ steps with probability $\geq\delta$, we can generate two independent random walks of length $\ell$, starting from $s$ and $t$ respectively, and detect if they collide, i.e., terminate at the same intermediate node. 
The critical observation is that using $\sqrt{n}$ walks from $s$ and $t$ gives $n$ potential collisions, which is sufficient to estimate probabilities on the order of $1/n$.
This argument draws from older ideas on using the \emph{birthday-paradox} in estimation problems~\cite{Motwani2007}.
A bidirectional algorithm for general graphs was first developed by Lofgren et al.~\cite{Lofgren2014} for PageRank estimation; the argument was subsequently simplified in \cite{lofgren2016personalized}, and extended to general Markov chains in \cite{banerjee2015fast}. 
Our work here further generalizes this line of work to computing single elements of powers of arbitrary matrices, with a particular application to solving for single elements in sparse linear systems.


\section{Solving Linear Equations via Series Approximation}
\label{sec:linsolve}

Unless specified otherwise, we use boldface letters (e.g. $\mathbf{x},\mathbf{y}$) to denote vectors in $\setR^{n\times 1}$, and capital letters (e.g. $A,Q$) to denote matrices in $\setR^{n\times n}$.


Given a linear system $\mathbf{y} = A\mathbf{x}$, where $A$ is a positive definite matrix, our aim is to estimate $\mathbf{x}[t]$ for some given index $t\in[n]$. 
This can be done by directly solving for $\mathbf{x}$; however, we are interested in settings where $n$ is very large, and hence direct solution techniques may be impractical. 

One approach for approximating the solution to $\mathbf{y} = A\mathbf{x}$ is to expand it via the Neumann series and then compute the leading terms of the summation.
In particular, if $A$ is positive definite, \nsedit{we can find $\gamma$ such that $G = I - \gamma A$ satisfies $\rho(G) < 1$. Then $\gamma\mathbf{y} = (I - (I - \gamma A)\mathbf{x}$  and we have a new system $\mathbf{x} = G\mathbf{x} + \mathbf{z}$ where $\mathbf{z} = \gamma \mathbf{y}$ and $G = I - \gamma A$. }
\nsedit{Since we ensure $\rho(G)<1$, we can write $\mathbf{x}$ as a von Neumann series: $\mathbf{x} = \sum_{k=0}^\infty G^k \mathbf{z}$.}
Thus to find the $t$ component of the solution vector $\mathbf{x}$, or $\mathbf{x}[t]$, we perform the operation $\ip{\mathbf{x}}{\mathbf{e}_t} = \sum_{k=0}^\infty \ip{G^k \mathbf{z}}{\mathbf{e}_t}$. 
Similar transformations have been used in \cite{dimov2015new, lee2014asynchronous, wu2016multi}.

\nsedit{Computing $\mathbf{x}[t]$ then amounts to computing $\pzlt := \ip{G^{\ell}\mathbf{z}}{\mathbf{e}_t} = \ip{\mathbf{z}}{(G^T)^{\ell}\mathbf{e}_t}$ for any $\ell$ and taking their sum for some $\ell \in \{0,\ldots,\lm\}$ where $\lm$ is a finite term truncating the power series.}
Let $Q:=G^T$; prior work by Banerjee and Lofgren \cite{banerjee2015fast} shows how to compute $\pzlt$ for the special case where $\mathbf{z}$ is a probability vector and $Q$ is a \emph{stochastic matrix} (i.e., with all nonnegative entries, and each row summing to $1$). 
We now extend this result for any $\mathbf{z}$ and a special class of matrices $Q$.

\nsedit{
Note that the error from truncating the series to $\lm$ can be determined a priori; using bounds for $\mathbf{z}$ and the condition that $G$ has spectral norm less than $1$, we can set $\lm \geq \frac{1}{\ln \rho(G)} \ln \left(\frac{\Delta(1 - \rho(G))}{||\mathbf{z}||} \right)$  to bound the series truncation error by $\Delta$.
\begin{proof}
Let $\varepsilon$ be the error from truncating the power series to $\lm$. Then by definition:
\begin{align*}
\varepsilon &= \left|\sum_{\ell=0}^\infty \ip{\mathbf{z}}{Q^\ell\mathbf{e}_t} - \sum_{\ell=0}^{\lm} \ip{\mathbf{z}}{Q^\ell\mathbf{e}_t} \right| \\
%&=\left| \sum_{\ell= \lm+1}^\infty \ip{\mathbf{z}}{Q^\ell\mathbf{e}_t} \right|\\
&= \left| \ip{\mathbf{z}}{ Q^{\lm + 1} \sum_{\ell= 0}^\infty Q^\ell\mathbf{e}_t} \right|\\
\end{align*} 
For any inner product of the form $\ip{\mathbf{z}}{Q^\ell \mathbf{e}_t}$, by the Cauchy - Schwarz inequality we have $|\ip{\mathbf{z}}{Q^\ell \mathbf{e}_t}| \leq ||\mathbf{z}|| \cdot ||{Q}^\ell \mathbf{e}_t|| \leq ||\mathbf{z}|| \rho(Q)^{\ell}$.
Hence we have
$\varepsilon \leq %||\mathbf{z}|| \rho(Q)^{\lm+1}\left( \sum_{\ell = 0}^{\infty}{\rho(Q)^{\ell}} \right) \\
 ||\mathbf{z}|| \rho(Q)^{\lm+1}\left(\frac{1}{1 - \rho(Q)} \right)$
Now suppose we want our error $\epsilon \leq \Delta$. Then we can set the upper bound for $\epsilon$ to $\Delta$ and solve for $\lm$.
%\begin{align*}
%||\mathbf{z}|| \rho(Q)^{\lm+1}\left(\frac{1}{1 - \rho(Q)} \right) =  \Delta \\
%(\lm + 1)\ln \rho(Q) = \ln \left(\frac{\Delta(1 - \rho(Q)) }{||\mathbf{z}||}\right) \\
%$\lm = \frac{1}{\ln \rho(Q)}  \ln \left(\frac{\Delta(1 - \rho(Q)) }{||\mathbf{z}||}\right) - 1$
Thus, if we take $\lm \geq \frac{1}{\ln \rho(Q)}  \ln \left(\frac{\Delta(1 - \rho(Q)) }{||\mathbf{z}||}\right)$, we will have an error of at most $\Delta$ when approximating the series.
\end{proof}
Since we can a priori bound the error resulting from truncating the von Neumann series to $\lm$, we will focus on the problem of estimating $\pzlt$. It is then straightforward to develop overall error bounds by combining the results of the additive truncation error $\Delta$ and the relative error guarantees we prove for our bidirectional algorithm in Theorem 1. 
}


%!TEX root = main.tex
\section{Existing Algorithms for Estimating Matrix Powers}
\label{sec:existing}

Based on the above discussion, we henceforth focus on developing bidirectional algorithms for estimating $\pzlt =\ip{\bz}{Q^{\ell}\mathbf{e}_t}$.
To do so, we first describe two existing algorithms that we use as primitives for our procedure: a forward MCMC technique based on the von Neumann-Ulam scheme~\cite{Wasow1952,ji2013convergence}, and a local variational method proposed by Andersen et al.~\cite{andersen2007local} for computing PageRank, and used by Lee et al.~\cite{lee2014asynchronous} in this setting. 
We present these along with their accuracy and running time guarantees -- some of these results follow directly from previous work (as we note in the appropriate sections), and we include their proofs here mainly for the sake of completeness.

We first introduce some notation which will help us better visualize the algorithm.
Drawing parallels to the case where $Q$ is a stochastic matrix and $\bz$ an element in the $n$-dimensional simplex (as in~\cite{banerjee2015fast}), we define a (weighted) directed graph $\G_Q(\V,\E)$ with states $\V = [n]$, and edges $(i,j)\in \E$ if $Q_{ij}\neq 0$. 
Each edge $(i,j)\in\E$ has an associated weight $w_{ij}\in\setR$, which we describe later.
We refer to the label for a node $v\in V$ (i.e., a dimension $v\in[n]$) as a \emph{dimension-index}, and the exponent of $Q$ as the \emph{step-index}.
Finally, we define $Q^+$ to denote the matrix with $Q^+_{ij} = |Q_{ij}|$. 
%Many of our bounds depend on the norm $||Q^+||_{\infty} = ||Q||_{\infty}$, i.e., the maximum (absolute) row-sum of $Q$
%; for ease of notation, we use the shorthand $\beta := ||Q||_{\infty}$. 


\subsection{Computing Matrix Powers via Iterative Local-Update}
\label{ssec:reverse}

One approach for estimating $\pzlt$ is via a standard power iteration for computing $Q^{\ell}\be_t$. 
In settings where $n$ is large such that direct power iteration is infeasible, one can use a `local' power iteration, which corresponds to a natural dynamic programming update. Informally, the algorithm estimates $\pzlt$ by starting off with a mass of $1$ on dimension-index $t$, and then `pushing' this mass in reverse along the edges of graph $G_Q$. 

To describe this \texttt{REVERSE-LOCAL-UPDATE} algorithm, we first define a \texttt{REVERSE-PUSH} local iteration. 
Essentially, this is a local power-iteration for computing $Q^{\ell}\mathbf{e}_t$, which adaptively exploits any sparsity in the computation.
This operation was defined by Andersen et al. \cite{andersen2007local} and subsequently used as a primitive in \cite{banerjee2015fast, lee2014asynchronous}.
The critical associated invariant in Lemma \ref{lem:pushinvariant} was first shown for Personalized PageRank\cite{andersen2007local}; however, it holds more generally for any matrix $Q$ (as we show below; also see~\cite{lee2014asynchronous}).
 

For each step-index $k \in\{0,1,\ldots,\ell \}$, we store two vectors: the \emph{estimate vector} $\qest{k}$ and the \emph{residual vector} $\res{k}$.
We initialize all $\res{k},\qest{k}, k\in[\ell]$ to $0$ except for $\res{0}$, which we set to $\mathbf{e}_t$.
Now, given any dimension-index $v\in[n]$ and step-index $k \in[\ell]$, the \texttt{REVERSE-PUSH} operation iteratively updates these vectors as follows:
\begin{algorithm}[!ht]
\caption{\texttt{REVERSE-PUSH}$(t,v,k)$}
\label{alg:push}
\begin{algorithmic}[1]
\REQUIRE Matrix $Q$, estimates $\qest{k}$, residuals $\res{k},\res{k+1}$
\RETURN New estimates $\qestnew{k}$, residuals $\resnew{k}$ computed as:
\begin{align*}
	\qestnew{k} &\leftarrow \qest{k} + \ip{\res{k}}{\mathbf{e}_v}\mathbf{e}_v \\
	\resnew{k} &\leftarrow \res{k} - \ip{\res{k}}{\mathbf{e}_v}\mathbf{e}_v \\
	\resnew{k+1} &\leftarrow \res{k+1} + \ip{\res{k}}{\mathbf{e}_v}(Q\mathbf{e}_v)
\end{align*}	
\end{algorithmic}
\end{algorithm}    

%The \texttt{REVERSE-PUSH} iteration results in the following critical invariant for the estimate and residual vectors:
\begin{lemma}
\label{lem:pushinvariant}
\emph{Given the initialization described above, after any sequence of \texttt{REVERSE-PUSH} operations, and for any $\bz\in\setR^n$ and $\ell\geq 0$, the estimates $\{\mathbf{q}_t^k\}$ and residuals $\{\mathbf{r}_t^k\}$  satisfy the following invariant:
\begin{align*}
\mathbf{p}_\bz^{\ell}[t] &= \left< \bz, \mathbf{q}_t^{\ell} \right> + \sum_{k=0}^{\ell} \left<\bz, Q^k\mathbf{r}_t^{\ell-k}\right> = \left< \bz, \mathbf{q}_t^{\ell}  + \sum_{k=0}^{\ell} Q^k\mathbf{r}_t^{\ell-k}\right>
\end{align*}
}
\end{lemma}

The above invariant was first stated in~\cite{andersen2007local} for the case of PageRank vectors. 
For the sake of completeness, we present a proof (adapted from~\cite{Lofgren2014}) for general matrices; an identical invariant was also presented in~\cite{lee2014asynchronous}.

\begin{proof}
For our chosen initialization (i.e., $\res{0} = \mathbf{e}_t$, and all other estimate and residual vectors set to $0$), the invariant simplifies to $\mathbf{p}_\bz^{\ell}[t] = \ip{\bz }{Q^{\ell}\mathbf{e}_t}$ which is true by definition.
Now, assuming the invariant holds at any stage with vectors $\{\qest{k}\}, \{\res{k}\}_{k\in[\ell]}$, let $\{\qestnew{k}\}, \{\resnew{k}\}_{k\in[\ell]}$ be the new vectors after executing a \texttt{REVERSE-PUSH}$(t,v,k)$ operation for any given $k\in[\ell]$ and $\bz\in\setR^n$. We define:
$$\Delta_v^{k} = \left( \mathbf{\tilde{q}}_t^{\ell} + \sum_{i=0}^{\ell}  (Q^i)\resnew{\ell-i} \right)- \left( \qest{\ell} + \sum_{i=0}^{\ell}  (Q^i) \mathbf{r}_t^{\ell-i}\right)$$
Now to show that the invariant holds following \texttt{REVERSE-PUSH}$(t,v,k)$, it suffices to show that $\Delta_v^{k}$ is zero for any $v\in V$ and $k\in[\ell]$. 

We now have three cases: $(i)$ if $\ell < k$, then the \texttt{REVERSE-PUSH}$(t,v,k)$ operation does not affect the residual or estimate vectors $\{\mathbf{q}_t^{i},\mathbf{r}_t^{i}\}_{i<k}$, and hence 
$\Delta_v^k=0$;
$(ii)$ If $\ell = k$, we have:
\begin{align*}
\Delta_v^{k} &= \left( \mathbf{\tilde{q}}_t^{k} +  \mathbf{\tilde{r}}_t^{k}\right)- \left( \mathbf{q}_t^{k} + \mathbf{r}_t^{k}\right)\\
&= \mathbf{q}_t^{k} + \left<\mathbf{r}_t^{k}, \mathbf{e}_v \right>\mathbf{e}_v +  \mathbf{r}_t^{k} - \left<\mathbf{r}_t^{k}, \mathbf{e}_v \right>\mathbf{e}_v - \mathbf{q}_t^{k} - \mathbf{r}_t^{k}
= 0
\end{align*}
$(iii)$ Finally, when $\ell > k$, we have: 
\begin{align*}
\Delta_v^k &= Q^{\ell-k}\left( \mathbf{\tilde{r}}_t^k - \mathbf{r}_t^k \right) + Q^{\ell-k-1}\left( \mathbf{\tilde{r}}_t^{k+1} - \mathbf{r}_t^{k+1} \right)\\
&= -\left<\mathbf{r}_t^k, \mathbf{e}_v \right>Q^{\ell-k}\mathbf{e}_v  + \left<\mathbf{r}_t^k, \mathbf{e}_v \right>Q^{\ell-k-1}\left(Q\mathbf{e}_v \right)
%&= -\left<\mathbf{r}_t^k, \mathbf{e}_v \right>\mathbf{e}_v(Q^T)^{\ell-k} + \left<\mathbf{r}_t^k, \mathbf{e}_v \right>\mathbf{e}_v (Q^T)^{\ell-k} 
= 0
\end{align*}
Hence we have shown that the invariant is preserved for any sequence of reverse push operations.
\end{proof}

The above invariant gives a natural iterative algorithm for computing $\pzlt$: perform repeated \texttt{REVERSE-PUSH} operations controlling the residual vectors $\mathbf{r}_t^k$, and use $\pzltest = \langle\bz,\mathbf{q}_t^{\ell}\rangle$ as the estimate. 
Depending on the norm we choose to control, we can get a bound for the error via H{\"o}lder's inequality.
In particular, placing an upper bound on the infinity norm (i.e., the maximum absolute value of the residual vectors) of some chosen $\delta_r>0$ gives us the error bounds: $|\pzlt - \langle\bz,\mathbf{q}_t^{\ell}\rangle|\leq ||x||_1\delta_r||Q||_{\infty}^{\ell}$~\footnote{This approach was used in~\cite{andersen2007local,Lofgren2014,banerjee2015fast}; an alternative is to control $||r_t^k||_2$ giving error bounds in terms of $||x||_2$, which was suggested in ~\cite{lee2014asynchronous}.}.

\begin{algorithm}[!ht]
\caption{\texttt{REVERSE-LOCAL-UPDATE}$(t,Q, \ell, \delta_r)$}
\label{alg:rwork}
\begin{algorithmic}[1]
\REQUIRE Matrix $Q$, maximum step-index $\ell$, target residual threshold $\delta_r$
\STATE{Initialize all residual $\res{k}$ and estimate vectors $\qest{k}, k \in[\ell]$ to $0$; set $\res{0} = \mathbf{e}_t$}
\FOR{ $k \in \{0, 1, 2, ... \ell\}$ }
\WHILE{$\exists v$ such that $\left|\res{k}[v]\right| > \delta_r $}
\STATE{\texttt{REVERSE-PUSH}$(t,v,k)$}
\ENDWHILE
\ENDFOR
\RETURN $\{\qest{k}\}, \{\res{k}\}_{k \in[\ell]}$
\end{algorithmic}
\end{algorithm}    

Finally, we want to bound the running time of \texttt{REVERSE-LOCAL-UPDATE}$(t,Q,\ell,\delta_r)$. 
It is easy to see that in the worst case, the running time can be as much as the $\ell$-hop in-neighborhood of $t$ in $Q$. 
However, for a \emph{uniform random} choice of $t$, we can obtain a more informative bound. 
%Recall we define $\beta =  ||Q||_{\infty}$. Now we have the following:

\begin{lemma}
\label{lem:pushruntime}
\emph{For any $Q\in\setR^{n\times n}$ and uniform random dimension-index $t\in[n]$, the expected running time of \texttt{REVERSE-LOCAL-UPDATE}$(t,Q,\ell,\delta_r)$ is 
$$O\left(\frac{nnz(Q)}{n\delta_r}(\ell+1)||Q||_{\infty}^{\ell}\right)$$ 
}
\end{lemma}

In particular, note that if $||Q||_{\infty}\leq 1$ and $\ell=O(1)$, then the average running time is $O\left(\frac{nnz(Q)}{n\delta_r}\right)$.

\begin{proof}
Let $T(t)$ be the running time of \texttt{REVERSE-LOCAL-UPDATE}$(t,Q,\ell,\delta_r)$.
Recall we define $Q^+$ as the matrix with $Q^+_{ij} = |Q_{ij}|$; let $\hat{T}(t)$ be the running time of \texttt{REVERSE-LOCAL-UPDATE}$(t,Q^+,\ell,\delta_r)$. 
Then we have that for every matrix $Q$ and every $t$, we have $\hat{T}(t)>T(t)$ -- this follows from the fact that any cancellation between positive and negative residuals in \texttt{REVERSE-LOCAL-UPDATE}$(t,Q^+,\ell,\delta_r)$ can only decrease the number of iterations. 
Also, note that under $Q^+$ all residuals are positive, so we have for any $k\leq\ell,v\in[n]$, the residuals satisfy $\mathbf{r}_t^{k}[v]\leq \left(\mathbf{e}_v^TQ^\ell\right)[t]$.

Now let $d_i := \sum_j\mathds{1}_{\{Q_{ij}\neq 0\}}$, i.e., the support of $i^{th}$ row in $Q$, and $\mathbf{r}_t^{k}$ denote the residuals under \texttt{REVERSE-LOCAL-UPDATE}$(t,Q^+,\ell,\delta_r)$. 
%Recall we define $\beta = \max\{1,||Q||_{\infty}\}$.
From Algorithm \ref{alg:rwork}, we have $\hat{T}(t) = \sum_{k=0}^{\ell}\sum_{v\in[n]}\mathds{1}_{\mathbf{r}_t^{k}[v]>\delta_r}$.
Thus, the expected running time over a uniform random choice of $t\in[n]$ is given by
\begin{align*}
\frac{1}{n}\sum_t\hat{T}(t) &= \frac{1}{n}\sum_{t\in[n]}\sum_{k=0}^{\ell}\sum_{v\in[n]}\mathds{1}_{\{\mathbf{r}_t^{k}>\delta_r\}}d_v\\
&=\frac{1}{n}\sum_{k=0}^{\ell}\sum_{v\in[n]}\sum_{t\in[n]}\mathds{1}_{\{\mathbf{r}_t^{k}>\delta_r\}}d_v\\
&\leq\frac{1}{n}\sum_{k=0}^{\ell}\sum_{v\in[n]} \mathds{1}_{\{\left(\mathbf{e}_w^T(Q^+)^k\right)[t]>\delta_r\}}d_v\\
&=\frac{1}{n}\sum_{k=0}^{\ell}\sum_{v\in[n]} \frac{||\mathbf{e}_w^T(Q^+)^k||_1}{\delta_r}d_v\\
&\leq\frac{1}{n}\sum_{k=0}^{\ell}\sum_{v\in[n]} \frac{||Q^+||_{\infty}^k}{\delta_r}d_v\\
&\leq\left(\ell+1\right)||Q||_{\infty}^{\ell} \frac{nnz(Q)}{n\delta_r}
\end{align*}
\end{proof}


\subsection{Computing Matrix Powers via MCMC Sampling}
\label{ssec:forwardwork}


In the previous section, we computed $\pzlt$ by working backwards from $t$. 
Note that our final algorithm is independent of $\bz$.
We now present an alternate technique which is based on a forward MCMC sampling technique called the von Neumann-Ulam scheme. 
In this case, the algorithm starts from $\bz$, and computes $\pzlt$ for all $t\in[n]$.

More generally, given any vectors $\mathbf{a}$ and $\mathbf{b}$ and matrix $Q$, the von Neumann-Ulam scheme can be used for computing $\ip{\mathbf{a}}{Q^{\ell}\mathbf{b}}$.
To understand the algorithm, note that we can expand $\ip{\mathbf{a}}{Q^{\ell}\mathbf{b}}$ as the sum $\sum_{(v_0,\ldots, v_{\ell}) \in V^{\ell}}\left(\prod_{j \in [{\ell}]} Q_{v_{j-1}v_{j}} \right) \mathbf{a}[v_0] \mathbf{b}[v_{\ell}]$. 
Now we can interpret this sum as an expectation over an ${\ell}$-step random walk $W = (V_0,V_1,\ldots,V_{\ell})$ on $\G$, specified as follows:
\begin{itemize}
\item $V_0$, the starting node for the random walk, is sampled from $\sigma_{\mathbf{a}} = \{|\mathbf{a}[i]|/||\mathbf{a}||_1\}_{i\in [n]}$, and has an associated weight $w_{V_0} = sgn(\mathbf{a}[V_0])||\mathbf{a}||_1$.
\item The transition probability matrix for the walk is given by $P_{ij} = \{|Q_{ij}|/||Q_i||_1\}$ (where $||Q_i||_1$ is the $1$-norm of the $i^{th}$ row of $Q$).
\item Each edge $(i,j)\in E$ has associated weight $w_{ij} = sgn(Q_{ij})||Q_i||_1$. 
\item The `score' for a walk $W$ is the product of weights of traversed edges, i.e.,
$$S_t^{\ell}(W) = w_{V_0}\prod_{i=0}^{{\ell}-1}w_{V_iV_{i+1}}\mathbf{b}[V_{\ell}]$$
\end{itemize}
%We can then sample paths according to $P$ and weight each of these products by the normalization factors we used to construct $P$ and $\sigma$ from $Q$ and $\bz$. There are two aspects to normalizing $Q$ and $\bz$, the sign, and ensuring row sums are equal to $1$. By normalizing the substochastic matrix $Q$ to get the stochastic $P$, we are removing a dummy state $A$ that absorbed probability for $Q$. 
%We construct $\mathbf{p}_A$ to store this leftover probability mass. 
%We sample from the distribution $\sigma (P)^k$ and then weight by the normalization factors to compute $\sum_{(v_0,\ldots, v_k) \in V^k}\left(\prod_{j \in [k]} Q_{v_{j-1}v_{j}} \right) \bz[v_0] \mathbf{r}_t^{\ell-k}[v_k]$ as an expectation. 
This procedure is summarized in Algorithm \ref{alg:fwalk}. 

\begin{algorithm}[ht]
\caption{\texttt{MCMC-SAMPLER}$(Q, {\ell}, \mathbf{a}, \mathbf{b})$}
\label{alg:fwalk}
\begin{algorithmic}[1]
\REQUIRE Matrix $Q$, exponent ${\ell}$, vectors $\mathbf{a}$ and $\mathbf{b}$
\STATE{Construct transition matrix $P_{ij} = \{|Q_{ij}|/||Q_i||_1\}$ and starting measure $\sigma_a = \{|\mathbf{a}[i]|/||\mathbf{a}||_1\}_{i\in V}$}
\STATE{Define node weights $w_{V_0} = sgn(\mathbf{a}[V_0])||\mathbf{a}||_1$ and edge weights $w_{V^iV^j} = sgn(Q_{V^iV^j})||Q_i||_1$}
\STATE{Construct source distribution $\sigma_{\mathbf{a}}$ with $\sigma_{\mathbf{a}}[i] = \frac{\bz[i]}{||\bz||_1}$}
\STATE{Sample $V^0 \sim \sigma_a$, and generate a random walk $W = \{V^0, V^1,\ldots, V^{\ell}\}$ of length ${\ell}$ on $\G$  using transition probability $P$.}
\RETURN Walk-score $S_{t}^{\ell}(W) = w_{V_0}\prod_{i=0}^{{\ell}-1}w_{V_iV_{i+1}}\mathbf{b}[V_{\ell}]$
\end{algorithmic}
\end{algorithm} 

%Once we have the sampled scores for $n_f$ walks each of length $l$, we can combine the estimate $\ip{\bz}{\qest{\ell}}$ with an average over the scores $\{S_{t, i}^{\ell}\}$ from to compute the overall estimator $\pzltest$.
%Again, as noted in the section describing the forward walk estimation, provided that $Q$ is substochastic with spectral norm less than $1$, we can find an induced transition matrix that is fully stochastic as well as the described normalization terms $\pabsorb{\ell}$.

%Recall we define $\beta = ||Q||_{\infty}$. 
Now we have the following Lemma.
\begin{lemma}
\label{lem:mcmc}	
\emph{
\texttt{MCMC-SAMPLER}$(Q, {\ell},\mathbf{a},\mathbf{b})$ returns a walk-score $S_{t}^{\ell}(W)$ which satisfies:
\begin{enumerate}
\item $\mathbb{E}_{W \sim \sigma_{\mathbf{a}}(P)^{\ell}} \left[S_{t}^{\ell}(W)\right] = \left<\mathbf{a}, Q^{\ell}\mathbf{b} \right>$
\item $S_{t}^{\ell}(W) \in [\;-||Q||_{\infty}^{\ell}||\mathbf{a}||_1||\mathbf{b}||_{\infty} ,\; ||Q||_{\infty}^{\ell}||\mathbf{a}||_1||\mathbf{b}||_{\infty} \;]$
\end{enumerate}
}
\end{lemma}


\begin{proof}
We can expand $\mathbb{E}_{W \sim \sigma_{\mathbf{a}}(P)^{\ell}} \left[S_{t}^{\ell}(W)\right]$ to obtain:
\small
\begin{align*}
&\mathbb{E}_{W \sim \sigma_{\mathbf{a}}(P)^{\ell}} \left[S_{t}^{\ell}(W)\right] \\ 
&= \sum_{(V^0, ... V^{\ell}) \in [n]^{\ell+1}} \left(\frac{sgn(\mathbf{a}[V_0])||\mathbf{a}||_1|\mathbf{a}[V^0]|}{||\mathbf{a}||_1} \right. \\
&\left. \times \left( \prod_{j \in [0, \ell-1]} \frac{sgn(Q_{V^jV^{j+1}}||Q_{V^j}||_1)|Q_{V^jV^{j+1}}|}{||Q_{V^j}||_1} \right) \mathbf{b}[V^{\ell}] \right) \\
& = \sum_{(V^0, ... V^{k}) \in [n]^{\ell+1}} \left(\mathbf{a}[V_0] \left( \prod_{j \in [0,\ell-1]} Q_{V^jV^{j+1}}\right) \mathbf{b}[V^{\ell}] \right)
\end{align*}
\normalsize
The final expression is exactly the definition of $\ip{\mathbf{a}}{Q^{\ell}\mathbf{b}}$; hence $\mathbb{E}_{W \sim \sigma_{\mathbf{a}}(P)^{\ell}} \left[S_{t}^{\ell}(W)\right] = \ip{\mathbf{a}}{Q^{\ell}\mathbf{b}}$.


Next, in order to bound the score $S_{t}^{\ell}(W)$, recall that
\begin{align*}
S_{t}^{\ell}(W) = w_{V_0}\prod_{i=0}^{\ell-1}w_{V_iV_{i+1}}\mathbf{b}[V_{\ell}]
\end{align*}

We defined $w_{V_0} = sgn(\mathbf{a}[V_0])||\mathbf{a}||_1$, so trivially $|w_{V_0}| \leq ||\mathbf{a}||_1$.
Additionally, $w_{V_iV_{i+1}} = sgn(Q_{V^iV^j})||Q_i||_1 $ and by definition $||Q_i||_1  \leq ||Q||_{\infty}$ and $| \mathbf{b}[V^{\ell}] |\leq ||\mathbf{b}||_\infty$.
Hence, since we are multiplying ${\ell}$ edge scores, we finally obtain
$|S_{t}^{\ell}(W)| \leq ||Q||_{\infty}^{\ell}||\mathbf{a}||_1||\mathbf{b}||_\infty$ as stated in the lemma.
\end{proof}



%!TEX root = main.tex

\section{A Bidirectional Algorithm for Computing Matrix Powers}
\label{ssec:bidiralgo}

Finally, we present our main contribution: a bidirectional estimator for $\pzlt = \langle\bz,Q^{\ell}\be_t\rangle$. 
Our algorithm follows the general structure proposed by Lofgren et al.~\cite{Lofgren2014,banerjee2015fast} for PageRank and Markov Chain transition probability estimation.
It comprises of two distinct components: first we use  \texttt{REVERSE-LOCAL-UPDATE} to estimate approximate values of $\left(Q^{\ell}\be_t\right)[i]$ for all steps $\ell \in[\lm]$ and $i\in[n]$. 
We then use \texttt{MCMC-SAMPLER} to reduce the error in these estimates to get our desired accuracy.


Intuitively, the advantage we gain from combining the two previous algorithms is that running a small amount of local-update reduces the variance in the walk-scores significantly, which then allows us to perform sampling more effectively.
More specifically, given a desired error threshold $\delta$, we first run \texttt{REVERSE-LOCAL-UPDATE}$(t,Q,\ell,\delta_r)$ for an appropriately chosen $\delta_r\gg\delta$ (cf. Theorem \ref{thm:main}). 
At this point, from Lemma \ref{lem:pushinvariant}, we know that we have
$\pzlt = \ip{\bz}{\mathbf{q}_t^{\ell}} + \sum_{k=0}^{\ell} \ip{\bz}{Q^k\mathbf{r}_t^{\ell-k}}$, with $\mathbf{r}_t^{k}[v]\leq\delta_r$ for all $v\in[n],k\leq\ell$.
Now, instead of ignoring the residual terms (as done in~\cite{andersen2007local,lee2014asynchronous}; cf. Section \ref{ssec:reverse}), we can further reduce our error by using \texttt{MCMC-SAMPLER}$(Q,k,\bz,\mathbf{r}_t^k)$ to estimate the residual $\langle\bz,\mathbf{r}_t^k\rangle$ for all $k\leq\ell$.
Note however that the resulting error is better than ignoring the residual, and also better than directly executing \texttt{MCMC-SAMPLER}$(Q,\ell,\bz,\be_{t})$, as the residuals have much smaller magnitude than $1$, and hence the walk-scores have lower variance.
\begin{algorithm}[ht]
\caption{\texttt{BIDIR-MATRIX-POWER}$(Q, \bz, t,\lm)$}
\label{alg:linearsysest}
\begin{algorithmic}[1]
%\REQUIRE Matrix $Q$, source vector $\bz$ defining the system $\mathbf{x} = G \mathbf{x} + \bz$, target $t$, number of random walks $n_f$, series truncation parameter $\lm$.
\REQUIRE Matrix $Q$, vector $\bz$, target node-index $t$.
\STATE{Compute estimate and residual vectors via $\{\res{\ell}\}, \{\qest{\ell}\}$ = \texttt{REVERSE-LOCAL-UPDATE}$(t,Q, \lm, \delta_r)$}
\FOR{ $l \in \{1, 2,\ldots, \lm \}$}
\FOR{ $i \in [n_f]$}
\STATE{$k \sim Unif[0, \ell]$}
\STATE{$S_{i,t}^\ell$ = \texttt{MCMC-SAMPLER} $(Q, k, \bz, \ell \cdot \res{\ell - k})$}
\ENDFOR
\STATE{$\pzlt = \ip{\bz}{\qest{\ell}} + \frac{1}{n_f} \sum_{i=0}^{n_f}S_{i,t}^\ell$}
\ENDFOR
\RETURN $\sum_{\ell=0}^{\lm} \pzlt$ 
\end{algorithmic}
\end{algorithm} 

\begin{lemma}
\texttt{BIDIR-MATRIX-POWER} computes an unbiased estimator of $\pzlt$
\begin{align*}
\pzlt &=  \left<\bz, Q^l \mathbf{e}_t \right>= \left<\bz, \mathbf{q}_t^{\ell} \right> + \sum_{k=0}^{\ell} \left<\bz , Q^k\mathbf{r}_t^{\ell-k}\right>\\
&= \mathbb{E}\left[\hat{\mathbf{p}}_\bz^{\ell}[t] \right]
\end{align*}
\end{lemma}

\begin{proof}
By definition of the estimator $\pzltest$ and linearity of the expectation operator:
\begin{align*}
\mathbb{E}\left[\pzltest\right] &= \left<\bz, \mathbf{q}_t^{\ell} \right> + \frac{1}{n_f}\sum_{i=1}^{n_f} \mathbb{E}_{k \sim Unif[0, l]}\left[S_{t}^{k}\right] \\
&= \left<\bz, \mathbf{q}_t^{\ell} \right> + \mathbb{E}_{k \sim Unif[0, \ell]}\left[S_{t}^{k}\right]
\end{align*}

From Lemma 2, we know $\mathbb{E}\left[S_{t}^{k}\right] = \ip{\mathbf{a}}{Q^k \mathbf{b}}$ when we call \texttt{MCMC-SAMPLER}$(Q, k, \mathbf{a}, \mathbf{b})$, so in the linear system estimator, we obtain: 
\begin{align*}
\mathbb{E}\left[\pzltest \right] &= \left<\bz, \mathbf{q}_t^{\ell} \right> + \mathbb{E}_{k \sim Unif[0, \ell]} \left[\ip{\bz}{Q^k ( \ell \cdot \mathbf{r}_t^{\ell-k})} \right] \\
&= \left<\bz, \mathbf{q}_t^{\ell} \right> + \frac{1}{\ell} \sum_{k = 0}^\ell\ip{\bz}{Q^k ( l \cdot \mathbf{r}_t^{\ell-k} )}\\
&= \left<\bz, \mathbf{q}_t^{\ell} \right> + \sum_{k = 0}^l\ip{\bz}{Q^k\mathbf{r}_t^{\ell-k}}
\end{align*}
Recall from Lemma 1, that for any matrix $Q$ and after any sequence of reverse push operations, $\mathbf{p}_\bz^{\ell}[t] = \left<\bz, Q^{\ell} \mathbf{e}_t \right>$ obeys the invariant: 
$$\pzlt = \left<\bz, \mathbf{q}_t^{\ell} \right> + \sum_{k=0}^{\ell} \left<\bz, Q^k\mathbf{r}_t^{\ell-k}\right>$$
Hence $\mathbb{E}\left[\pzltest \right] = \pzlt$.
\end{proof}


\begin{theorem}
\label{thm:main}
Theorem: LINEAR-SYSTEM-ESTIMATOR estimates $\mathbf{x}[t]$ with relative accuracy $\epsilon$ and with probability $1-p_{fail}$ in running time 
$$ O\left( \left(\frac{\lm^2 |\mathbf{z}|_1 nnz(Q)}{\epsilon \delta n} \right)^{2/3} \left(\ln \frac{\lm}{p_{fail}} \right)^{1/3} \right)$$

\end{theorem}
\begin{proof}
We will prove this statement by first considering the single estimator $\pzltest$ for some $\ell$ since $\ip{\mathbf{x}_t}{\mathbf{e}_t} = \sum_{l=0}^{\infty} \pzlt$.

\end{proof}
We will prove this statement by first considering the single estimator $\pzltest$ for some $\ell$ since $\ip{\mathbf{x}_t}{\mathbf{e}_t} = \sum_{l=0}^{\infty} \pzlt$.

Consider the estimator $S_{t}^{\ell}$. We have already shown that $\EE\left[S_{t, i}^{\ell}\right] = \pzlt - \left< \sigma, \qest{\ell} \right>$ and computed the work to achieve relative error $\epsilon$ for these estimators. Now observe:

\begin{align*}
&\mathbb{P}\left[ | \pzltest - \pzlt| \geq  \epsilon \pzlt \right] \\
&\leq \mathbb{P}\left[ |X -\mathbb{E}[X]| \geq \epsilon \mathbb{E}[X]\right] \leq p_{fail}
\end{align*}
By Lemma 5, the work done by \texttt{MCMC-SAMPLER} to achieve relative accuracy $\epsilon$ with probability $1-p_{fail}$ is 
$$ {O}\left( \frac{\lm^2 |\mathbf{z}|_1^2 \delta_r^2 \beta^{2\lm}}{\epsilon^2\delta^2}\ln \left(\frac{\lm}{p_{fail}}\right)\right) $$

By Lemma 6, we get that the forward and reverse work running times are equal asymptotically if we set 
$$\delta_r = \sqrt[3]{\frac{nnz(Q) \epsilon^2\delta^2}{n\lm |\mathbf{z}|_1^2\beta^{2\lm}\ln(\lm/p_{fail}) } }$$
Substituting in this value for the forward walk work and we obtain:

$$ O\left( \left(\frac{\lm^2 |\mathbf{z}|_1 nnz(Q)}{\epsilon \delta n} \right)^{2/3} \left(\ln \frac{\lm}{p_{fail}} \right)^{1/3} \right)$$
which is in fact the total running time.
\begin{lemma}
The work done by \texttt{MCMC-SAMPLER} with averaged score variables $\frac{1}{n_f}\sum_{i=0}^{n_f} S_{i, t}$ returns an estimator $\widehat{\EE[S_{i, t}^l]}$ with relative accuracy $\epsilon$ with probability $1-p_{fail}$ in running time
$${O}\left( \frac{\lm^2 |\mathbf{z}|_1^2 \delta_r^2 \beta^{2\lm}}{\epsilon^2\delta^2}\ln \left(\frac{\lm}{p_{fail}}\right)\right)$$
provided that $q$
\end{lemma}

\begin{proof}
Let $X_i = S_{t, i}^{\ell}$ and $X = \sum_{i=1}^n X_i$. Then $X_i \in [-\lm|\mathbf{z}|_1\delta_r \beta^{\lm}, \lm|\mathbf{z}|_1\delta_r \beta^{\lm}]$ by Lemma 3. Since $\EE[X_i] \geq \delta$, we let $c = \lm |\mathbf{z}|_1\delta_r \beta^{\lm}$, $a = \delta$ and by Lemma 6
$$n_f \geq \frac{2 \lm^2|\mathbf{z}|_1^2 \delta_r^2 \beta^{2\lm}}{\epsilon^2\delta^2}\ln \left(\frac{\lm}{p_{fail}}\right)$$
random walks to ensure that $\mathbb{P}\left[ | \widehat{\EE[S_{i, t}^l]}  -  \EE[S_{i, t}^l]|  \geq  \epsilon  \EE[S_{i, t}^l] \right] \leq p_{fail}$ holds for $\lm$ sets of $\pzltest$.
Hence, the total work in the forward estimate to ensure 
$\mathbb{P}\left[ |X -  \EE[X]]|  \geq  \epsilon  \EE[X] \right] \leq p_{fail}$ for $\lm$ sets of estimators $X$ is
$${O}\left(\frac{\lm^2 \delta_r^2|\mathbf{z}|_1^2 \beta^{2\lm}}{\epsilon^2\delta^2}\ln \left(\frac{\lm}{p_{fail}}\right)\right)$$
\end{proof}

\begin{lemma}
Set 
$$\delta_r = \sqrt[3]{\frac{nnz(Q)(\lm + 1) \epsilon^2\delta^2}{n\lm^2 |\mathbf{z}|_1^2\beta^{2\lm - 1}\ln(\lm/p_{fail}) } }$$
to balance the reverse push and forward walk work.
\end{lemma}
\begin{proof}
From Lemma 2, the work from the reverse push operation is
$O\left(\frac{nnz(Q)}{n\delta_r}(\lm+1)\beta\right)$.
To get the optimal running time asymptotically, we set the forward work and reverse work equal to each other and solve for $\delta_r$:
\[\frac{nnz(Q)}{n\delta_r}(\lm+1)\beta = \frac{\lm^2 \delta_r^2|\mathbf{z}|_1^2 \beta^{2\lm}}{\epsilon^2\delta^2}\ln \left(\frac{\lm}{p_{fail}}\right) \]
Thus we obtain
\begin{equation*}
\delta_r = \sqrt[3]{\frac{nnz(Q)(\lm + 1) \epsilon^2\delta^2}{\lm^2 n|\mathbf{z}|_1^2\beta^{2\lm -1}\ln(\lm/p_{fail}) } }
\end{equation*}
For $\lm$ sufficiently large, we can replace $\lm + 1$ with just $\lm$ and $\beta^{2\lm -1}$ with $\beta^{2\lm}$.
\end{proof}



\begin{lemma}
Set $\lm = \frac{1}{\ln \rho(G)} \ln \left(\frac{\delta(1 - \rho(G))}{||\mathbf{z}||} \right)$  to satisfy additive error threshold of $\delta$.
\end{lemma}
\begin{proof}
Note that $\lm$ controls error by truncating the power series $\sum_{l=0}^\infty \ip{\mathbf{z}}{Q^l\mathbf{e}_t}$.
Suppose we have this error as $\Delta =\sum_{l=0}^\infty \ip{\mathbf{z}}{Q^l\mathbf{e}_t} - \sum_{l=0}^{\lm} \ip{\mathbf{z}}{Q^l\mathbf{e}_t} = \ip{\mathbf{z}}{Q^{\lm} \sum_{l=1}^\infty Q^{\ell} \mathbf{e}_t}$.
Then $\Delta(\lm) \leq ||\mathbf{z}|| \frac{\rho(G)^{\lm}}{1 - \rho(G)}$.
If we want additive error $\delta$ (that is, $ \delta \leq \Delta(\lm)$), provided $\delta \leq ||\mathbf{z}||$, we have
$\lm \geq \frac{1}{\ln \rho(G)} \ln \left(\frac{\delta(1 - \rho(G))}{||\mathbf{z}||} \right)$.
Recall that $\rho(G) < 1$, so $\ln \rho(G) < 0$ and the suggested value for $\lm$ increases as $\delta$ shrinks.
\end{proof}

\begin{lemma} [Hoeffding's Inequality]
Let $\{X_i\}$ be independent random variable s.t. for all $i$, $X_i \in [-c, c]$ a.s., and $|\mathbb{E}[X_i]| \geq a$. 
Then $X = \sum_{i= 1}^n X_i$ satisfies
$\mathbb{P}[|X - \mathbb{E}[X]| \geq \epsilon \mathbb{E}[X] ] \leq p_{fail}$ provided that
$$n \geq \frac{2c^2}{\epsilon^2 a^2}\ln\left(\frac{2}{p_{fail}} \right)$$
\end{lemma}

\begin{proof}
Let $X = \sum_{i=1}^n X_i$.  Since $E[X] = nE[X_i]$, $E[X] \geq n \epsilon a$. Let $t = \epsilon a$.
\begin{align*}
\mathbb{P}\left[|X - \mathbb{E}[X]| \geq \epsilon \mathbb{E}[X] \right] 
&\leq \mathbb{P}\left[|X - \mathbb{E}[X]| \geq nt \right] \\
&= \mathbb{P}\left[\left|\frac{1}{n}X - \frac{1}{n}\mathbb{E}[X]\right| \geq t \right]  
\end{align*}
Applying Hoeffding's inequality to the rightmost term above, we obtain
  \[\mathbb{P}\left[|X - \mathbb{E}[X]| \geq \epsilon \mathbb{E}[X] \right] \leq 2\exp \left(-\frac{2n^2t^2}{\sum_{i=1}^n(b_i - a_i)^2}\right) \]
  Hence substituting the values for $t$ and $b_i = c$, $a_i = -c$ gives us:
\[\mathbb{P}\left[|X - \mathbb{E}[X]| \geq \epsilon \mathbb{E}[X] \right] \leq 2\exp \left(-\frac{2n^2\epsilon^2a^2}{n(4c^2)}\right) \]
Now set this upperbound to $p_{fail}$ to obtain the relation:
$$ \frac{n \epsilon^2 a^2}{2c^2} = \ln\left(\frac{2}{p_{fail}} \right) $$
Thus, for $n \geq \frac{2c^2}{\epsilon^2 a^2}\ln\left(\frac{2}{p_{fail}} \right)$, we are guaranteed that $\mathbb{P}\left[|X - \mathbb{E}[X]| \geq \epsilon \mathbb{E}[X] \right] \leq p_{fail}$.

Note that if we require the failure conditions to hold for $\lm$ sets of $X's$, then we perform a simple union bound which adds a $\lm$ in the $\frac{2}{p_{fail}}$ term.
\end{proof} 

\begin{comment}
\subsection{Useful Tail Inequalities}
Suppose $X_i$ are i.i.d with $\mathbb{E}[X_i]=\mu$, $Var(X_i)=\sigma^2$ and $X_i\in [a,b]$. Then:
\begin{itemize}
\item Hoeffding's Inequality:
\begin{equation*}
\mathbb{P}\left[\left|\left(\frac{1}{n}\sum_{i=1}^nX_i\right)-\mu \right|\geq\epsilon|\mu|\right]\leq
2\exp\left(\frac{-2n\epsilon^2\mu^2}{(b-a)^2}\right)
\end{equation*}
\item Bernstein's Inequality: 
\begin{equation*}
\mathbb{P}\left[\left|\left(\frac{1}{n}\sum_{i=1}^nX_i\right)-\mu\right|\geq\epsilon|\mu|\right]\leq
2\exp\left(\frac{-n\epsilon^2\mu^2}{2\sigma^2+\frac{2}{3}\epsilon|\mu|(b-a)}\right)
\end{equation*}
\end{itemize}

\end{comment}

\bibliographystyle{ieeetr}
\bibliography{FastLinsolve_Refs} 

\end{document}